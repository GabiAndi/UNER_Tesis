\chapter{Tecnología y software utilizados}

\pagestyle{empty}

\section*{Descripción del capítulo}

En este capítulo se hablan de las tecnologías utilizadas para llevar a cabo el proyecto. Se trato siempre de utilizar tecnologías opensource y openhard, no solo para utilizarlas sin limitaciones de licencia o acuerdos, sino también por la cantidad de documentación y apoyo que reciben las mismas. \\
	
A lo largo del desarrollo del proyecto y del prototipo se utilizaron muchas alternativas a lo que normalmente se utiliza en industria, esto se debe a las tecnologías no son tan conocidas o ya se encuentran funcionando en la actualidad sistemas ya probados y de fácil manejo de marcas muy conocidas (ABB, Siemens, Kuka, etc.).

\newpage

\pagestyle{fancy}

\section{Modelado y simulación}
	
\subsection{FluidFlow}

\begin{figure}[H]
    \centering
    \includegraphics[scale=.5]{src/imagenes/tecnologias/fluidflow.png}
\end{figure}

FLUIDFLOW es un software para el cálculo de pérdidas de carga en redes unidimensionales (tuberías). A través de su interfaz gráfico de usuario GUI, se puede construir de forma muy rápida la red indicando de manera interactiva las características de los diferentes elementos que la componen: tuberías, válvulas, condiciones de contorno, filtros, etc. \\

Dispone de una extensa base de datos tanto de fluidos, materiales, equipos, etc. El usuario puede customizar a su interés dicha base de datos. \\

Calcula presiones, flujos, temperaturas, transferencias de calor, pérdidas lineales y singulares de presión de tuberías, velocidad del fluido en cualquier punto del sistema para flujos de una y dos fases. Basados en los principios y ecuaciones de mecánica de fluidos más reconocidos universalmente: número de Reynolds, Bernoulli, ecuaciones de Darcy Weisbach, diagrama de Moody, Hazen Williams, Duxbury, Wilson Addie Selgren y Clift entre otros.

\subsection{Autodesk Inventor}

\begin{figure}[H]
    \centering
    \includegraphics[scale=.5]{src/imagenes/tecnologias/inventor.png}
\end{figure}

\href{https://latinoamerica.autodesk.com/products/inventor/}{Autodesk Inventor} es un paquete de modelado paramétrico de sólidos en 3D producido por la empresa de software AutoDesk. Compite con otros programas de diseño asistido por computadora como SolidWorks, Pro/ENGINEER, CATIA y Solid Edge. Entró en el mercado en 1999, muchos años después que los antes mencionados y se agregó a las Series de Diseño Mecánico de Autodesk como una respuesta de la empresa a la creciente migración de su base de clientes de diseño mecánico en dos dimensiones hacia la competencia, permitiendo que los computadoras personales ordinarias puedan construir y probar montajes de modelos extensos y complejos. \\

Como modelador paramétrico, no debe ser confundido con los programas tradicionales de CAD. Inventor se utiliza en diseño de ingeniería para producir y perfeccionar productos nuevos, mientras que en programas como Autocad se conducen solo las dimensiones. Un modelador paramétrico permite modelar la geometría, dimensión y material de manera que si se alteran las dimensiones, la geometría actualiza automáticamente basándose en las nuevas dimensiones. Esto permite que el diseñador almacene sus conocimientos de cálculo dentro del modelo, a diferencia del modelado no paramétrico, que está más relacionado con un "tablero de bocetos digitales". Inventor también tiene herramientas para la creación de piezas metálicas. \\

Para este proyecto en concreto se utilizó la licencia educativa que provee AutoDesk desde su página.

\subsection{Matlab y Simulink}

\begin{figure}[H]
    \centering
    \includegraphics[scale=.6]{src/imagenes/tecnologias/matlab.png}
\end{figure}

\href{https://la.mathworks.com/products/matlab.html}{MATLAB} es un sistema de cómputo numérico que ofrece un entorno de desarrollo integrado (IDE) con un lenguaje de programación propio (lenguaje M). Está disponible para las plataformas Unix, Windows, macOS y GNU/Linux. Entre sus prestaciones básicas se hallan la manipulación de matrices, la representación de datos y funciones, la implementación de algoritmos, la creación de interfaces de usuario (GUI) y la comunicación con programas en otros lenguajes y con otros dispositivos hardware. El paquete MATLAB dispone de dos herramientas adicionales que expanden sus prestaciones, a saber, Simulink (plataforma de simulación multidominio) y GUIDE (editor de interfaces de usuario - GUI). Además, se pueden ampliar las capacidades de MATLAB con las cajas de herramientas (toolboxes); y las de Simulink con los paquetes de bloques (blocksets). \\

Es un software muy usado en universidades y centros de investigación y desarrollo. En los últimos años ha aumentado el número de prestaciones, como la de programar directamente procesadores digitales de señal o crear código VHDL. \\

En 2004, se estimaba que MATLAB era empleado por más de un millón de personas en ámbitos académicos y empresariales. \\

\textbf{Simulink} es un entorno de programación visual, que funciona sobre el entorno de programación Matlab. Es un entorno de programación de más alto nivel de abstracción que el lenguaje interpretado Matlab. Simulink viene a ser una herramienta de simulación de modelos o sistemas, con cierto grado de abstracción de los fenómenos físicos involucrados en los mismos.
Se hace hincapié en el análisis de sucesos, a través de la concepción de sistemas (cajas negras que realizan alguna operación). Es ampliamente usado en ingeniería electrónica en temas relacionados con el procesamiento digital de señales (DSP), involucrando temas específicos de ingeniería biomédica, telecomunicaciones, entre otros. También es muy utilizado en ingeniería de control y robótica. \\

Para este proyecto se utilizo una licencia educativa de Matlab que permite utilizar todas las funciones incorporadas.

\section{Desarrollo de software}

\subsection{Git}

\begin{figure}[H]
    \centering
    \includegraphics[scale=.1]{src/imagenes/tecnologias/git.jpeg}
\end{figure}

\href{https://git-scm.com/}{Git} es un software de control de versiones diseñado por Linus Torvalds, pensando en la eficiencia, la confiabilidad y compatibilidad del mantenimiento de versiones de aplicaciones cuando estas tienen un gran número de archivos de código fuente. Su propósito es llevar registro de los cambios en archivos de computadora incluyendo coordinar el trabajo que varias personas realizan sobre archivos compartidos en un repositorio de código. \\

Al principio, Git se pensó como un motor de bajo nivel sobre el cual otros pudieran escribir la interfaz de usuario o front end como Cogito o StGit. Sin embargo Git se ha convertido desde entonces en un sistema de control de versiones con funcionalidad plena. Hay algunos proyectos de mucha relevancia que ya usan Git, en particular, el grupo de programación del núcleo Linux. \\

El mantenimiento del software Git está actualmente (2009) supervisado por Junio Hamano, quien recibe contribuciones al código de alrededor de 280 programadores. En cuanto a derechos de autor Git es un software libre distribuible bajo los términos de la versión 2 de la Licencia Pública General de GNU.

\subsection{GitHub}

\begin{figure}[H]
    \centering
    \includegraphics[scale=.1]{src/imagenes/tecnologias/github.png}
\end{figure}

\href{https://github.com/}{GitHub} es una forja (plataforma de desarrollo colaborativo) para alojar proyectos utilizando el sistema de control de versiones Git. Se utiliza principalmente para la creación de código fuente. El software que opera GitHub fue escrito en Ruby on Rails. Desde enero de 2010, GitHub opera bajo el nombre de GitHub, Inc. Anteriormente era conocida como Logical Awesome LLC. El código de los proyectos alojados en GitHub se almacena típicamente de forma pública. \\

El 4 de junio de 2018 Microsoft compró GitHub por la cantidad de 7500 millones de dólares, al inicio el cambio de propietario generó preocupaciones y la salida de algunos proyectos de este repositorio, sin embargo no fueron representativos. GitHub continúa siendo la plataforma más importante de colaboración para proyectos Open Source.

\subsection{Qt}

\begin{figure}[H]
    \centering
    \includegraphics[scale=.1]{src/imagenes/tecnologias/qt.png}
\end{figure}

\href{https://www.qt.io/}{Qt} es un framework multiplataforma orientado a objetos ampliamente usado para desarrollar software que utilicen interfaz gráfica de usuario, así como también diferentes tipos de herramientas para la línea de comandos y consolas para servidores que no necesitan una interfaz gráfica de usuario. \\

Qt es desarrollada como un software libre y de código abierto a travez de Qt Project, donde participa tanto la comunidad, como desarrolladores de Nokia, Digia y otras empresas. Qt es distribuida bajo los términos de GNU Lesser General Public License y otras. Por otro lado, Digia está a cargo de las licencias comerciales de Qt desde marzo de 2011.

\subsection{Raspberry Pi OS}

\begin{figure}[H]
    \centering
    \includegraphics[scale=.5]{src/imagenes/tecnologias/rpios.png}
\end{figure}

\href{https://www.raspberrypi.org/software/}{Raspberry Pi OS} (anteriormente llamado Raspbian) es una distribución del sistema operativo GNU/Linux basado en Debian, y por lo tanto libre para la SBC Raspberry Pi, orientado a la enseñanza de informática. El lanzamiento inicial fue en junio de 2012. Desde 2015, la Raspberry Pi Foundation lo ha proporcionado de forma oficial como el sistema operativo primario para la familia de placas SBC de Raspberry Pi. Hay varias versiones de Raspbian, siendo la actual Raspbian Buster. \\

Técnicamente el sistema operativo es un port no oficial de Debian armhf para el procesador (CPU) de Raspberry Pi, con soporte optimizado para cálculos en coma flotante por hardware, lo que permite dar más rendimiento en según qué casos. El port fue necesario al no haber versión Debian armhf para la CPU ARMv6 que contiene el Raspberry Pi. \\

Al ser una distribución de GNU/Linux las posibilidades son infinitas. Todo software de código abierto puede ser recompilado en la propia Raspberry Pi para arquitectura armhf que pueda utilizarse en el propio dispositivo en caso de que el desarrollador no proporcione una versión ya compilada para esta arquitectura. Además esta distribución, como la mayoría, contiene repositorios donde el usuario puede descargar multitud de programas como si se tratase de una distribución de GNU/Linux para equipos de escritorio. Todo esto hace de Raspberry Pi un dispositivo que además de servir como placa con microcontrolador clásica, tenga mucha de la funcionalidad de un ordenador personal. Lo que lo puede convertir en una alternativa a los ordenadores personales, especialmente para personas con pocos recursos, para la extensión de la informática en países subdesarrollados o para aplicaciones que no soliciten muchos requerimientos.

\section{Gestión}

\subsection{Open SSH}

\begin{figure}[H]
    \centering
    \includegraphics[scale=.4]{src/imagenes/tecnologias/ssh.png}
\end{figure}

\href{https://www.openssh.com/}{OpenSSH} es la principal herramienta de conectividad para el inicio de sesión remoto con el protocolo SSH. Cifra todo el tráfico para eliminar las escuchas, el secuestro de conexiones y otros ataques. Además, OpenSSH proporciona un amplio conjunto de capacidades de tunelización segura, varios métodos de autenticación y opciones de configuración sofisticadas. \\

SSH (o Secure SHell) es el nombre de un protocolo y del programa que lo implementa cuya principal función es el acceso remoto a un servidor por medio de un canal seguro en el que toda la información está cifrada. Además de la conexión a otros dispositivos, SSH permite copiar datos de forma segura (tanto archivos sueltos como simular sesiones FTP cifradas), gestionar claves RSA para no escribir contraseñas al conectar a los dispositivos y pasar los datos de cualquier otra aplicación por un canal seguro tunelizado mediante SSH, también puede redirigir el tráfico del (Sistema de Ventanas X) para poder ejecutar programas gráficos remotamente. El puerto TCP asignado es el 22. \\

SSH trabaja de forma similar a como se hace con telnet. La diferencia principal es que SSH usa técnicas de cifrado que hacen que la información que viaja por el medio de comunicación vaya de manera no legible, evitando que terceras personas puedan descubrir el usuario y contraseña de la conexión ni lo que se escribe durante toda la sesión.

\subsection{Grafana}

\begin{figure}[H]
    \centering
    \includegraphics[scale=.5]{src/imagenes/tecnologias/grafana.png}
\end{figure}

\href{https://grafana.com/}{Grafana} es una herramienta hecha en software libre, específicamente con licencia Apache 2.0, ideada por Torkel Ödegaard (quien todavía está al frente de su desarrollo y mantenimiento) y creada en enero de 2014. Este desarrollador sueco comenzó su carrera en el ambiente .NET y en 2012 (hasta la fecha) sigue ofreciendo servicios de desarrollo y consultoría en esta popular plataforma privativa, de forma paralela con el desarrollo de software libre. \\

Grafana está escrita en Lenguaje Go (creado por Google) y Node.js LTS y con una fuerte Interfaz de Programación de Aplicaciones (API); es una aplicación que ha venido escalando posiciones, con una comunidad entusiasta de más de 600 colaboradores bien integrados (son 7 desarrolladores líderes -Torkel a la cabeza- y 5 a tiempo parcial para poder coordinar tal grupo de personas). Su código fuente está publicado en GitHub. \\

Grafana es una herramienta para visualizar datos de serie temporales. A partir de una serie de datos recolectados obtendremos un panorama gráfico de la situación de una empresa u organización.

\subsection{OpenVPN}

\begin{figure}[H]
    \centering
    \includegraphics[scale=.3]{src/imagenes/tecnologias/openVPN.png}
\end{figure}

\href{https://openvpn.net/}{OpenVPN} es una herramienta de conectividad basada en software libre: SSL (Secure Sockets Layer), VPN Virtual Private Network (red virtual privada). OpenVPN ofrece conectividad punto-a-punto con validación jerárquica de usuarios y host conectados remotamente. Resulta una muy buena opción en tecnologías Wi-Fi (redes inalámbricas IEEE 802.11) y soporta una amplia configuración, entre ellas balanceo de cargas. Está publicado bajo la licencia GPL, de software libre. \\

Es una herramienta multiplataforma que ha simplificado la configuración de VPN's frente a otras más antiguas y difíciles de configurar como IPsec y haciéndola más accesible para gente inexperta en este tipo de tecnología. \\

Las VPN se usan generalmente para:

\begin{itemize}
    \item{Conexión entre diversos puntos de una organización a través de Internet.}
    \item{Conexiones de trabajadores domésticos o de campo con IP dinámicas.}
    \item{Soluciones extranet para clientes u organizaciones asociadas con los cuales se necesita intercambiar cierta información en forma privada pero no se les debe dar acceso al resto de la red interna.}
    \item{Además brinda una excelente fiabilidad en la comunicación de usuarios móviles así como también al unir dos puntos distantes como agencias de una empresa dentro de una sola red unificada.}
\end{itemize}

\subsection{InfluxDB}

\begin{figure}[H]
    \centering
    \includegraphics[scale=.5]{src/imagenes/tecnologias/influxDB.png}
\end{figure}

\href{https://www.influxdata.com/}{InfluxDB} es una base de datos de series de tiempo de código abierto (TSDB) desarrollada por InfluxData. Está escrito en Go y optimizado para el almacenamiento y la recuperación rápidos y de alta disponibilidad de datos de series de tiempo en campos como el monitoreo de operaciones, métricas de aplicaciones, datos de sensores de Internet de las cosas y análisis en tiempo real. \\

InfluxDB no tiene dependencias externas y proporciona un lenguaje similar a SQL, escuchando en el puerto 8086, con funciones integradas centradas en el tiempo para consultar una estructura de datos compuesta de medidas, series y puntos. Cada punto consta de varios pares clave-valor denominados fieldset y una marca de tiempo. Cuando se agrupan por un conjunto de pares clave-valor llamado conjunto de etiquetas, estos definen una serie. Finalmente, las series se agrupan mediante un identificador de cadena para formar una medida. \\

Los valores pueden ser enteros de 64 bits, puntos flotantes de 64 bits, cadenas y valores booleanos. Los puntos se indexan por su tiempo y conjunto de etiquetas. Las políticas de retención se definen en una medición y controlan cómo se reducen y eliminan los datos. Las consultas continuas se ejecutan periódicamente y almacenan los resultados en una medición objetivo.

\subsection{Telegraf}

\begin{figure}[H]
    \centering
    \includegraphics[scale=.35]{src/imagenes/tecnologias/telegraf1.png}
\end{figure}

\href{https://www.influxdata.com/time-series-platform/telegraf/}{Telegraf} es un agente de servidor basado en complementos para recopilar y enviar métricas y eventos de bases de datos, sistemas y sensores de IoT. \\

Telegraf está escrito en Go y se compila en un solo binario sin dependencias externas, ademas requiere una huella de memoria mínima. \\

Telegraf recopila y envía todo tipo de datos:

\begin{itemize}
    \item{\textbf{Base de datos:} se conecta a fuentes de datos como MongoDB, MySQL, Redis y otras para recopilar y enviar métricas.}
    \item{\textbf{Sistemas:} recopila métricas de su pila moderna de plataformas, contenedores y orquestadores en la nube.}
    \item{\textbf{Sensores de IoT:} recopila datos de estado críticos (niveles de presión, niveles de temperatura, etc.) de los sensores y dispositivos de IoT.}
\end{itemize}

\textbf{Agente:} \\

Telegraf puede recopilar métricas de una amplia gama de entradas y escribirlas en una amplia gama de salidas. Está impulsado por complementos tanto para la recopilación como para la salida de datos, por lo que es fácilmente ampliable. Está escrito en Go, lo que significa que es un binario compilado e independiente que se puede ejecutar en cualquier sistema sin necesidad de dependencias externas, sin necesidad de npm, pip, gem u otras herramientas de administración de paquetes. \\

\textbf{Cobertura:} \\

Posee más de 200 complementos ya escritos por expertos en la materia sobre los datos de la comunidad, es fácil comenzar a recopilar métricas de sus puntos finales. Aún mejor, la facilidad de desarrollo de complementos significa que puede crear su propio complemento para adaptarse a sus necesidades de monitoreo. Incluso puede usar Telegraf para analizar los formatos de datos de entrada en métricas. Estos incluyen: Protocolo de línea InfluxDB, JSON, Graphite, Value, Nagios y Collectd. \\

\textbf{Flexible:} \\

La arquitectura del complemento Telegraf es compatible con sus procesos y no lo obliga a cambiar sus flujos de trabajo para trabajar con la tecnología. Ya sea que lo necesite para sentarse en el borde o de manera centralizada, simplemente se adapta a su arquitectura en lugar de al revés. La flexibilidad de Telegraf hace que sea una decisión fácil de implementar.

\begin{figure}[H]
    \centering
    \includegraphics[scale=.3]{src/imagenes/tecnologias/telegraf2.png}
\end{figure}

\section{Hardware de control}

\subsection{Raspberry Pi}

\begin{figure}[H]
    \centering
    \includegraphics[scale=.3]{src/imagenes/tecnologias/rpi.png}
\end{figure}

\href{https://www.raspberrypi.org/}{La Raspberry Pi} es una serie de ordenadores de placa reducida, ordenadores de placa única u ordenadores de placa simple (SBC) de bajo coste desarrollado en el Reino Unido por la Raspberry Pi Foundation, con el objetivo de poner en manos de las personas de todo el mundo el poder de la informática y la creación digital. Aunque no se indica expresamente si es hardware libre o con derechos de marca, en su web oficial explican que disponen de contratos de distribución y venta con dos empresas, pero al mismo tiempo cualquiera puede convertirse en revendedor o redistribuidor de las tarjetas Raspberry Pi, por lo que da a entender que es un producto con propiedad registrada, manteniendo el control de la plataforma, pero permitiendo su uso libre tanto a nivel educativo como particular. \\

En cambio, el software sí es de código abierto, siendo su sistema operativo oficial una versión adaptada de Debian, denominada Raspberry Pi OS, aunque permite usar otros sistemas operativos, incluido una versión de Windows 10. En todas sus versiones, incluye un procesador Broadcom, memoria RAM, GPU, puertos USB, HDMI, Ethernet (el primer modelo no lo tenía), 40 pines GPIO (desde la Raspberry Pi 2) y un conector para cámara. Ninguna de sus ediciones incluye memoria, siendo esta en su primera versión una tarjeta SD y en ediciones posteriores una tarjeta MicroSD.