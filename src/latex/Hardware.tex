\chapter{Diseño del hardware}

\pagestyle{empty}

\newpage

\pagestyle{fancy}

\section{Variador de frecuencia}
	
El variador de frecuencia es la solución eficaz para mejorar la eficiencia energética y reducir el consumo de energía entre un 20\% y un 70\%. Regula la velocidad de motores eléctricos para que la electricidad que llega al motor se ajuste a la demanda real de la aplicación. \\

Un variador de frecuencia por definición es un regulador industrial que se encuentra entre la alimentación energética y el motor. La energía de la red pasa por el variador y regula la energía antes de que ésta llegue al motor para luego ajustar la frecuencia y la tensión en función de los requisitos del procedimiento. \\

El uso de variadores de frecuencia para el control inteligente de los motores tiene muchas ventajas financieras, operativas y medioambientales ya que supone una mejora de la productividad, incrementa la eficiencia energética y a la vez alarga la vida útil de los equipos, previniendo el deterioro y evitando paradas inesperadas que provocan descensos de productividad. \\

Como elección de variador de velocidad, el \href{https://www.se.com/ar/es/product/ATV630D75M3/variador-de-velocidad-altivar-process-atv600-atv630-75kw-100-hp-200-240-v-ip00/?filter=business-1-automatizaci\%C3\%B3n-y-control-industrial\&parent-subcategory-id=86129\&range=62317-altivar-process-600\&selected-node-id=12644445003}{Altivar Process Atv600 Atv630 75Kw 100 Hp} de Schneider Electric se adapta bien a los requisitos de potencia impuestos por los motores de los sopladores. Dispone de un gran número de opciones de configuración y de control, permitiendo una ampliación de sus características en el futuro.

\begin{figure}[H]
    \centering
    \includegraphics[scale=.14]{src/imagenes/hardware/variador.jpg}
    \caption{Altivar Process Atv600 Atv630 75Kw 100 Hp.}
\end{figure}

\section{Sensores}

En la pileta de lodos activos, actualmente se están sensando los siguientes parámetros:

\begin{itemize}
    \item{Temperatura.}
    \item{PH.}
    \item{Oxígeno disuelto.}
\end{itemize}

\subsection{Nivel de lodo}

Para poder tener un mayor control en el caudal de aire requerido, es necesario añadir un sensor de nivel. Esto permitirá determinar la edad del lodo y estimar cada cuanto es necesario aumentar la población de microorganismos.

\section{Controlador de aireación}

Esta etapa es la mas importante. En ella se encuentra el hardware de control conformado por una Raspberry Pi 3 B+ y electrónica transductora que se encarga de convertir los tipos de señales. \\

La microcomputadora es la encargada de procesar los datos de los sensores y enviar una respuesta al variador de frecuencia, para así poder controlar la velocidad de los sopladores. Además se encarga de almacenar y enviar los datos de telemetría de todo el proceso al servidor de métricas, para así poder visualizar la información en tiempo real o en un futuro realizar un análisis detallado. \\

Provee de una interfaz de control mediante red, para poder configurar todos sus parámetros y realizar modificaciones en su funcionamiento desde cualquier dispositivo. \\

Corre un sistema operativo basado en Debian para la arquitectura ARMv8, esto significa que posee todas las características de una computadora de propósito general ejecutando los siguientes procesos:

\begin{itemize}
    \item{Software controlador del proceso.}
    \item{Servidor de administración SSH.}
    \item{Servidor de administración HMI.}
    \item{Cliente OpenVPN.}
    \item{Cliente de base de datos influxdb.}
\end{itemize}

\begin{figure}[H]
    \centering
    \includegraphics[scale=.45]{src/imagenes/hardware/controlador.png}
    \caption{Diagrama de conexiones del controlador.}
\end{figure}

Todas las configuraciones realizadas al software que ejecuta la Raspberry Pi son explicadas en el capitulo de software. Por ahora solo diremos que el controlador del proceso y el servidor HMI son software desarrollados por nosotros y están disponibles en GitHub. El resto es software libre que utilizamos y configuramos según la documentación disponible oficialmente.

\section{Servidor de métricas}

Para esta etapa se requiere una computadora de escritorio con un almacenamiento moderado. En ella se correrá un servicio de métricas con el Software de Grafana. \\

Los requerimientos son los siguientes:

\begin{itemize}
    \item{8GB de RAM (o más).}
    \item{2TB de espacio de almacenamiento (si es mas mejor).}
    \item{Procesador de 4 núcleos a 3.6GHz.}
    \item{Sistema operativo GNU/Linux.}
    \item{Conexión de red Gigabite o superior (por el volumen de datos).}
\end{itemize}

Grafana es de código abierto y permite visualizar todos los datos generados desde el proceso de una manera muy visual y relativamente sencilla. Ademas provee de alertas mediante mails o indicadores visuales, esto hace mas fácil la detección de valores fuera del rango normal de operación. \\

Todo esto se explica en el capítulo de software.

\section{Interfaz para el operario}

Proporciona una interfaz de usuario gráfica y fácil de utilizar, permite al operario la configuración de los parámetros del proceso desde prácticamente cualquier dispositivo conectado a internet. \\

Para una estación de control fija, es recomendable un monitor de por lo menos 1080p, para poder visualizar de manera cómoda el proceso.