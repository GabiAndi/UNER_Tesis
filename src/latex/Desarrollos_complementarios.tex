\chapter{Anexo II: Desarrollos complementarios}

\pagestyle{empty}

\newpage

\pagestyle{fancy}

\section{Proceso de tratamiento de efluentes}
\label{anexo:tratamientoDeEfluentes}

\subsection{Tratamiento preliminar}

El ingreso del efluente se da a través de una tubería que finaliza en una pileta de medición de caudal, la cual posee una medidor en V y un sensor de ultrasonido. Luego el líquido continua hacia el tanque pulmón, no sin antes pasar por un tornillo drenador de residuos finos.

\begin{figure}[H]
    \centering
    \includegraphics[scale=.4]{src/imagenes/tratamiento/caudal.jpg}
    \caption{Medición de caudal proveniente de la línea de MDF.}
\end{figure}

\begin{figure}[H]
    \centering
    \includegraphics[scale=.45]{src/imagenes/tratamiento/tornillo.png}
    \caption{Tornillo drenador de residuos finos.}
\end{figure}

El efluente llega por gravedad a través de la canaleta de ingreso, hasta una caja de hormigón que tiene amurado una chapa de acero inoxidable con forma de vertedero de 90° por donde desborda la corriente líquida. El efluente crudo rebosa por el vertedero atravesando un canasto totalmente perforado de acero inoxidable. Los orificios tienen un diámetro de 8 $mm$ y su función es de retener sólidos.

\begin{figure}[H]
    \centering
    \includegraphics[scale=.4]{src/imagenes/tratamiento/grilla.jpg}
    \caption{Canasto de acero inoxidable.}
\end{figure}

El filtro canasto esta colgado debajo del vertedero y posee un sistema de críquet para izamiento. De esta manera el operador procede a retirar los sólidos. \\

El efluente filtrado antes de llegar al tanque ecualizador pasa por una cámara desarenadora de 7 $m^{3}$, cuya función es actuar como sedimentador estático de las partículas que están contenidas en la corriente líquida. \\

A caudal de trabajo promedio de la planta, esto implica un tiempo de retención de 10 a 15 minutos, lo que permite separar arenas y pequeñas piedras provenientes de las purgas del chip wash. \\

A partir de aquí, el efluente pasa al tanque ecualizador cuya función es homogeneizar picos de caudal y concentración del efluente crudo. Esta construido en hormigón armado con un volumen útil de 100 $m^{3}$ de capacidad. Para mantener homogeneizado el volumen del tanque pulmón e impedir la decantación de sólidos, se instaló una grilla construida en caño de acero inoxidable AISI 304 con 20 orificios de salida de 25 $mm$ de diámetro y repartidos en todo el fondo de la cuba.

\begin{figure}[H]
    \centering
    \includegraphics[scale=.6]{src/imagenes/tratamiento/ecualizador.png}
    \caption{Ecualizador.}
\end{figure}

\begin{figure}[H]
    \centering
    \includegraphics[scale=.5]{src/imagenes/tratamiento/fondoEcualizador.png}
    \caption{Grilla en el fondo de la cuba.}
\end{figure}

\subsection{Tratamiento primario}

La unidad de tratamiento físico-químico y sedimentación, tiene como fin efectuar una remoción gruesa de DBO5, DQO y sólidos en suspensión. \\

El sedimentador primario para el tratamiento físico-químico, esta construido en hormigón armado, se ha diseñado y dimensionado para tratar un caudal de 15 $m^{3}$ por hora y picos de hasta 25 $m^{3}$ por hora. \\

El efluente crudo ingresa por una caja vertedero a 60°, para medición de caudales.
Allí recibe la dosificación del policloruro de Aluminio (PAC) y posteriormente la dosis de hidróxido de sodio (soda cáustica). Luego ingresa a la primer cámara de reacción llamada flash-mix.

\begin{figure}[H]
    \centering
    \includegraphics[scale=.45]{src/imagenes/tratamiento/sodaCaustica.jpg}
    \caption{Añadido de soda caustica.}
\end{figure}

El tipo, la concentración y la secuencia de las especies químicas a dosificar están relacionados con las características del efluente, y su adecuada dosificación optimiza la formación del floculos y la clarificación. El PH óptimo de trabajo es 6,4, y las dosis de los reactivos surgen en función de los estudios de jar test realizados oportunamente. Como su nombre lo indica, en esta cámara flash-mix se produce la mezcla íntima entre el efluente crudo a tratar y los reactivos químicos. La cámara flash-mix tiene un tiempo de retención variable de acuerdo al caudal de efluente tratado. Esta construida de hormigón armado con las siguientes dimensiones: $1.30m$ de largo, $1.30m$ de ancho, y unos $3,3m$ de profundidad, con un volumen útil de $5m^{3}$. Esta cámara está equipada con un agitador, donde existe la posibilidad de modificar las velocidades de rotación, de acuerdo al efluente a tratar.

\begin{figure}[H]
    \centering
    \includegraphics[scale=.7]{src/imagenes/tratamiento/agitador.png}
    \caption{Sona de agitación.}
\end{figure}

Existe una segunda cámara, idéntica a la anterior donde se dosifica el polímero. Ambas cámaras cuentan con un PHmetro indicador con dos puntos de alarma. \\

Luego de producirse en ambas cámaras los procesos de coagulación-floculación, el efluente con el flóculo desarrollado fluye por rebalse e ingresa a una cámara intermedia, luego por una cañería de 6 pulgadas de diámetro de acero inoxidable montada en el inferior de la cámara, pasa hacia la columna central del sedimentador de 8m de diámetro, donde se separan los sólidos suspendidos. \\

El sedimentador se construyó en hormigón armado y está diseñado a caudal de trabajo, con una carga hidráulica conservadora, a razón de $0,5\frac{m{3}}{hr \cdot m^{2}}$. Posee un tirante líquido de $2,8m$ y un volumen de $140m^{3}$ de capacidad. \\

A los caudales promedio de trabajo de la planta, el tiempo de retención es de 10 horas aproximadamente, con este tiempo se logra reducir al mínimo el arrastre de sólidos suspendidos. Los mismos se recolectan en la parte superior mediante un barredor de superficie, así como los decantados son barridos por el barredor de medio puente de accionamiento.

\begin{figure}[H]
    \centering
    \includegraphics[scale=.4]{src/imagenes/tratamiento/sedimentador.png}
    \caption{Sedimentador.}
\end{figure}

\subsection{Tratamiento de lodos}

\subsubsection{Espesamiento y deshidratación de lodos}

Los barros generados en el sedimentador, se purgan del sistema mediante válvula automática y se envían a un tanque de lodos de $10m^{3}$ de capacidad, donde se someten a un proceso de acondicionamiento final y posterior deshidratación en el filtro prensa. La concentración lograda del lodo oscila entre $6\%$ al $8\%$.

\begin{figure}[H]
    \centering
    \includegraphics[scale=.5]{src/imagenes/tratamiento/secador.png}
    \caption{Deshidratación de lodo.}
\end{figure}

\begin{figure}[H]
    \centering
    \includegraphics[scale=.5]{src/imagenes/tratamiento/tanqueDeLodo.png}
    \caption{Tanque de lodos.}
\end{figure}

\subsubsection{Filtro Prensa}

La función del filtro prensa es deshidratar los barros provenientes del tratamiento físico-químico en forma de tortas compactas reteniéndolas entre los marcos. El agua filtrada se colecta junto al rebose del decantador primario, en un foso colector de aguas claras desde donde son bombeadas al tratamiento biológico. \\

Consta de 50 placas móviles y 2 placas fijas de $1000mm \times 1000mm$. Las mismas tienen dos superficies filtrantes y las fijas sólo una. Las placas móviles llevan un juego de telas dobles que cubren cada una de las superficies filtrantes y las placas fijas una tela simple que cubren sus únicas superficies filtrantes. El material de las telas es polipropileno. En total se generan 50 tortas. Cada torta tiene un volumen de 33 litros. El filtro se mantiene cerrado a una presión de $300bar$ mediante la central hidráulica.

\begin{figure}[H]
    \centering
    \includegraphics[scale=.5]{src/imagenes/tratamiento/filtroPrensa.png}
    \caption{Filtro prensa.}
\end{figure}

\subsection{Tratamiento secundario}

\label{anexo:tratamientoSecundario}

Las aguas provenientes del tratamiento físico químico y el filtrado proveniente de la deshidratación de lodo se recolectan en el foso de aguas claras. Estos líquidos contienen materia orgánica, en gran medida en forma soluble, la cual deberá ser eliminada en el tratamiento secundario de tipo biológico. \\

Para el tratamiento del agua residual se tiene una unidad de lodos activados AIS (Advent Integral System) provista por el grupo EIMCO/ADVENT de U.S.A. La planta se puso en marcha en Julio del año 2000. \\

A diferencia de otros tipos de plantas de depuración biológica, el sistema AIS incorpora dentro de una misma unidad tres áreas: una pileta de aireación, una zona anóxica y el clarificador. Este sistema permite trabajar a una alta concentración de biomasa. Posee un volumen de trabajo útil de $1400m^{3}$ de capacidad. Con un tiempo de residencia en el reactor de 5-6 días. \\

Esta planta está diseñada para tratar un promedio diario de $12 \frac{m^{3}}{hora}$ de agua residual y es para reducir TSS, DBO, y DQO hasta niveles aceptables.

\begin{figure}[H]
    \centering
    \includegraphics[scale=.8]{src/imagenes/tratamiento/biologico.png}
    \caption{Pileta de tratamiento biológico.}
\end{figure}

Para comprender este proceso definimos como lodos activados, a una mezcla de microorganismos específicos y agua residual a depurar, aireados en forma permanente, durante un cierto tiempo (aprox. 3 a 4 días). Luego mediante un proceso de sedimentación se separa el lodo del agua depurada. \\

Debido a que el agua a tratar contiene niveles de NTK (Nitrógeno Total Kejdall) algo elevados, la planta cuenta con una zona anóxica para la desnitrificación. Esta etapa permite mantener los niveles entrantes de NTK por debajo de los límites inhibidores y evita los problemas de estabilización dentro del clarificador, causados por no controlar la desnitrificación. \\

Desde el tratamiento primario el agua residual ingresa en la porción anóxica de la unidad AIS. Esta zona se caracteriza por un bajo nivel de oxigeno disuelto. Mediante la bomba dosificadora se suministra fósforo como nutriente, para mejorar la actividad biológica. El agitador sumergido mantiene una agitación suave y constante, para impedir que el lodo activado sedimente en el fondo. \\

Desde la zona anóxica el lodo activado, pasa por vaso comunicante a la sección de aireación. Es aquí donde los microorganismos específicos degradan la mayor parte de la materia orgánica, utilizándola como nutriente para realizar su ciclo de reproducción y crecimiento, y como consecuencia realizan una depuración biológica de agua residual que es nuestro objetivo. \\

El aire necesario para este proceso es vital, se incorpora mediante dos sopladores de aire tipo roots, y se distribuye mediante una grilla de difusión por burbujas gruesas sujeta al fondo de la cuba de aireación. \\

La concentración de biomasa en el sistema se mantiene entre 6000 - 8000 mg/l. Mediante la bomba de elevación de aire (air lift), el lodo activado asciende desde la zona de aireación e ingresa al canal DTF (desaireación, transición, floculación), en este canal se brindan condiciones (baja velocidad) para que el lodo activado pierda el aire disuelto. Mediante la bomba se dosifica una solución de polímero catiónico para ayudar a la floculación, que tendrá lugar, minutos después dentro del clarificador.

\begin{figure}[H]
    \centering
    \includegraphics[scale=.45]{src/imagenes/tratamiento/plantaDeTratamientoBiologico.png}
    \caption{Gráfica del proceso de lodos activados.}
\end{figure}

Parte del caudal del lodo activado que fluye por el canal DTF, ingresa al clarificador y el resto se recicla a la zona anóxica, diluyendo la corriente ingresante desde el tratamiento primario. Estos caudales se regulan en forma manual, mediante la válvula tapón del clarificador y la válvula de compuerta de la zona anóxica. \\

La tubería de distribución del clarificador está especialmente diseñada para crear una velocidad descendente del lodo activado. Esta velocidad descendente inducida es una característica única patentada del diseño del AIS. El distribuidor es una tubería de acero inoxidable de gran diámetro con orificios en el fondo que tiene bocas sobre la parte superior de los orificios para dirigir el flujo descendente. \\

Desde la tubería de distribución del clarificador, el lodo es espesado en el fondo del cono y el efluente claro es recolectado por dos tuberías perforadas sumergidas en la parte superior del clarificador de manera de recolectar sólo el agua clarificada. Aquí se separa la biomasa del agua tratada. Es decir recuperamos los microorganismos para que continúen depurando mas agua residual. \\

El agua depurada es recolectada en una cuba de hormigón de $8m^{3}$ de capacidad. Las bombas centrifugas son las encargadas de bombear el agua tratada. Esta operación es automática, a través de boyas de control de nivel. El lodo activado concentrado en el fondo del clarificador cónico se recicla hacia la zona anóxica, mediante vasos comunicante por orificios desde las zonas cónicas hacia la zona anóxica. La purga de fondo necesaria para mantener la concentración de biomasa del sistema en los niveles adecuados, se realiza a través de una tubería de 3 pulgadas con válvula automática temporizada y controlada a través de un PLC. El tiempo de apertura de la purga se fija mediante ensayos de laboratorio.

\subsubsection{Caracteristicas constructivas}

\begin{itemize}
    \item{\textbf{Zona de aireación:} Tanque de aireación de $16.61m$ de largo, $8.74m$ de ancho y $7.00m$ de alto. El nivel de trabajo del liquido es de $6m$. El volumen total de aireación provisto es de $960m^{3}$.}
    \item{\textbf{Clarificador integral:} Las paredes del clarificador tienen un ángulo de 60° y esta fabricado en hormigón armado. El clarificador 		integral tiene $6.81m$ de largo y $8.74m$ de ancho, con un área de superficie de $59.5m^{2}$. La velocidad de carga hidráulica es de $0.2\frac{m^{3}}{hr \cdot m^{2}}$ a velocidad de flujo promedio $12\frac{m^{3}}{h}$, y $0.42\frac{m^{3}}{hr \cdot m^{2}}$ a velocidad de flujo de pico $25\frac{m^{3}}{h}$.}
    \item{\textbf{Zona anóxica:} Las dimensiones de este sector son de $2,78m$ de largo, $8,74m$ de ancho y $7,00 m$ de alto total. El nivel de trabajo del líquido es de 6 metros. El volumen total anóxico provisto es $240m^{3}$.}
    \item{\textbf{Cuba de recolección de efluente clarificado:} Construida de hormigón armado externamente y solidaria a la pileta principal. El agua desde la parte superior del clarificador, es colectada por dos tuberías sumergidas, ubicadas simétricamente y perforadas en la parte inferior. Antes de caer a la batea, el líquido es medido por un aforador en V, con ángulo de 30° construido en acero inoxidable.}
\end{itemize}

\subsection{Tratamiento terciario}

\subsubsection{Etapa de desinfección}

El efluente tratado en la etapa biológica será clorado para desinfección mediante solución de hipoclorito de sodio comercial al $10\% p/p$. Esta desinfección se realiza en la cuba N°1 de hormigón de $6m^{3}$ de capacidad, aprovechando la caída
del agua desde el vertedero de rebose para lograr una buena mezcla. \\

Para aumentar el tiempo de contacto, el agua pasa por rebose a una segunda cuba de hormigón de $6m^{3}$ de capacidad. De esta forma el tiempo de contacto agua-cloro será de unos 40 minutos para asegurar la desinfección antes del ingreso al sistema de filtros. \\

Contar con dos cubas permitirá eventualmente realizar limpiezas del fondo para extraer materia orgánica que sedimentará a raíz de la oxidación del hipoclorito. El sistema cuenta con un equipo de detención de cloro residual en línea con alarma de máximo y mínimo. La bomba se regula manualmente de forma de mantener un residual de cloro libre entre $0,1ppm$ a $0,2ppm$.

\subsubsection{Filtro de arena}

Luego del tratado con hipoclorito de sodio, el efluente circula a presión por un filtro de manto profundo de arena. \\

La bomba aspira desde la cuba N°2 y es la encargada de hacer circular el líquido a través del sistema terciario. En dicha cuba hay boyas para controlar el nivel (En caso de mínimo nivel se detiene la bomba). La presión normal de salida de dicha bomba oscila entre $5.2bar$ a $5.8bar$. El caudal a tratar se regula a la entrada del filtro de arena mediante la válvula mariposa. Este caudal se ajusta en función del rebalse del tratamiento biológico de manera de equilibrar el sistema y mantener períodos de filtración largos. Se debe evitar que la bomba trabaje en forma intermitente. \\

El filtro esta construido en chapa de acero al carbono, y revestido interiormente con epoxi bituminoso. El diámetro es de $1,00m$ y la altura de $2,60m$. \\

El lecho filtrante se compone básicamente por un manto soporte conformado por canto rodado de granulometrías diferentes (en nuestro caso 5 tamaños) y un lecho filtrante de arena (gruesa y fina). El sentido de flujo del agua a tratar es vertical de arriba hacia abajo. Este tipo de filtros, con mantos filtrantes de arena de $1.5m$ de espesor pueden lograr remociones del 95\% de partículas hasta de 3 micrones de tamaño (dependiendo de la calidad de agua de ingreso).

\subsubsection{Filtro de carbón activado}

El efluente filtrado y libre de sólidos en suspensión, finalmente circula por un equipo purificador de carbón activado para el pulido final de sólidos y eliminación de cloro residual. \\

El equipo consta de un tanque cilíndrico de chapa de acero de $1,3m$ de diámetro y $2,2m$ de alto. Posee un sistema colector interno de acero inoxidable sumergido en un manto soporte de canto rodado y sobre éste se deposita el manto de carbón. \\

El flujo de agua es descendente. Durante el funcionamiento, el lecho de carbón se va compactando con lo cual es necesario efectuar lavados contracorriente al menos cada 24 horas.

\section{Desarrollo matemático del sistema}

\section{Desarrollo del sistema de control}