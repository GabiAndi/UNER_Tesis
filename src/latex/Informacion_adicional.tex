\chapter*{Sobre el trabajo}
\addcontentsline{toc}{chapter}{Sobre el trabajo}

\pagestyle{empty}

\section*{Historia}
	
Este proyecto nace como un trabajo práctico de la cátedra \textit{Gestión ambiental}, a cargo de las docentes Paola Sinner y Norma Sanabria. Para el cual se abordó una problemática concreta de la industria de EGGER en Concordia. \\

En 2021 se realiza una visita educativa a la planta de tratamiento de efluentes de la empresa mencionada. Durante esta visita se detecta una problemática posible de ser abordada mediante el desarrollo de una solución mecatrónica. Surge así nuestro proyecto final de Ingeniería en Mecatrónica.

\section*{Agradecimientos}

Queremos agradecer principalmente a las docentes de la cátedra \textit{Gestión ambiental}, Paola Sinner y Norma Sanabria, quienes trajeron la problemática
e hicieron el nexo entre la industria y nosotros; además de apoyarnos en cada etapa del desarrollo, brindando información y realizando criticas constructivas. \\

Al profesor de la asignatura, Ignacio Terenzano, quien acompaño durante cada etapa del desarrollo, desde la selección del problema, el alcance y los objetivos esperados. Nos motivo a ir un paso mas allá y abordar el problema en conjunto, para realizar un desarrollo mucho mas completo y en profundidad. \\

A los docentes German Hachman, Juan Ramos, Daniel Gamero y Alcides Burna, quienes aportaron con datos y experiencia, para darle sustento al proyecto. \\

A los demás docentes y no docentes de la carrera, directivos de la facultad y administrativos. \\

Por último pero no menos importante a nuestras familias, quienes estuvieron desde siempre brindando apoyo y energía para que todo esto sea posible.