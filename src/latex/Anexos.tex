\chapter{Anexos}

\pagestyle{empty}

\newpage

\pagestyle{fancy}

\section{Motores de AC}

\subsection{Balance de potencias}

En un motor asíncrono existe una transformación de energía eléctrica en mecánica, que se transmite desde el estátor al rotor, a través del entrehierro, y el proceso de conversión está inevitablemente ligado con las pérdidas en los diferentes órganos de la máquina. Vamos a analizar el balance de energía que se produce en el funcionamiento del motor.

\begin{tcolorbox}[title=Balance de potencias]
    La potencia que la máquina absorbe de la red, si $V_{1}$ es la tensión aplicada por fase, $I_{1}$ la corriente por fase y $r_{1}$ el desfase entre ambas magnitudes, será:
    
    $$P_{1} = m_{1} \cdot V_{1} \cdot I_{1} \cdot \cos{(\phi_{1})}$$
\end{tcolorbox}

Esta potencia llega al estátor, y una parte se transforma en calor por efecto Joule en sus devanados, cuyo valor es:

$$P_{cu1} = m_{1} \cdot R_{1} \cdot I^{2}_{1}$$

La otra parte se pierde en el hierro: $P_{Fe1}$. La suma de ambas pérdidas representa la disipación total en el estátor $P_{p1}$:

$$P_{p1} = P_{cu1} \cdot P_{Fe1}$$

Como se tiene que las frecuencias de las corrientes en el rotor son muy reducidas, debido a que los deslizamientos en la máquina suelen ser pequeños (por ejemplo, para $s = 5\%$ con $f_{1} = 50 Hz$, resulta una $f_{2} = 2.5 Hz < f_{1}$), se considera entonces que prácticamente es el hierro del estátor el único origen de las pérdidas ferromagnéticas. De acuerdo con el circuito equivalente del motor de la Figura \ref{fgr:circuitoEquivalente2}, se podrá escribir:

$$P_{Fe} = P_{Fe_{1}} = m_{1} \cdot E_{1} \cdot I_{Fe} \approx m_{1} \cdot V_{1} \cdot I_{Fe}$$

La potencia electromagnética que llegará al rotor a través del entrehierro, y que denominaremos $P_{a}$ (potencia en el entrehierro), tendrá una magnitud:

$$P_{a} = P_{1} - P_{p1} = P_{1} - P_{cu1} - P_{Fe}$$

En el rotor aparecen unas pérdidas adicionales debidas al efecto Joule, $P_{cu2}$, y de valor:

$$P_{cu2} = m_{2} \cdot R_{2} \cdot I_{2}^{2} = m_{1} \cdot R_{2}^{'} \cdot I_{2}^{'2}$$

Las pérdidas en el hierro del rotor son despreciables debido al pequeño valor de $f_{2}$. La potencia que llegará al árbol de la máquina, denominada potencia mecánica interna, $P_{mi}$, será:

$$P_{mi} = P_{a} - P_{cu2}$$

Que teniendo en cuenta el significado de la resistencia de carga $R_{'c}$ del circuito equivalente, se podrá poner:

$$P_{mi} = m_{1} \cdot R_{2}^{'} \cdot \left( \frac{1}{s} - 1 \right) \cdot I_{2}^{'2}$$

La potencia útil en el eje será algo menor, debido a las pérdidas mecánicas por rozamiento y ventilación; denominando $P_{m}$ a estas pérdidas y $P_{u}$ a la potencia útil, resultará:

$$P_{u} = P_{mi} - P_{m}$$

En la Figura \ref{fgr:perdidas} se muestra, en la parte superior, el circuito equivalente exacto del motor y en la parte inferior un dibujo simplificado de la máquina. En cada caso se muestran, con flechas, las pérdidas que se producen en las diversas partes del motor. Es instructivo que el lector determine las potencias con el circuito equivalente y verifique claramente la situación de las mismas en la figura real. Obsérvese en ambos casos que se obtiene una potencia útil de salida a partir de una potencia de entrada $P_{1}$. El rendimiento del motor vendrá expresado por el cociente:

$$\nu = \frac{P_{u}}{P_{1}} = \frac{P_{u}}{P_{u} + P_{m} + P_{cu2} + P_{Fe} + P_{cu1}}$$

\begin{figure}[H]
    \centering
    \includegraphics[scale=.4]{src/imagenes/anexos/perdidas.png}
    \caption{Circuito equivalente exacto y distribución de las potencias en el motor.}
    \label{fgr:perdidas}
\end{figure}

\subsection{Par de rotación}

\begin{tcolorbox}[title=Balance de potencias]
    Si es $P_{u}$ la potencia mecánica útil desarrollada por el motor y $n$ la velocidad en r.p.m. a la que gira el rotor, el par útil $T$ (o torque) en N.m en el árbol de la máquina será el cociente entre $P_{u}$ y la velocidad angular de giro $\Omega = \frac{2 \cdot \pi \cdot n}{60}$, (en radianes mecánicos por segundo), y donde $n$ se expresa en r.p.m.:
    
    $$T = \frac{P_{u}}{\Omega} = \frac{Pu}{2 \cdot \pi \cdot \frac{n}{60}}$$
\end{tcolorbox}

De la definición de deslizamiento se deduce:

$$s = \frac{n_{1} - n}{n_{1}}$$

Y la expresión del par se convierte en:

$$T = \frac{Pa}{2 \cdot \pi \cdot \frac{n1 \cdot (1 - s)}{60}}$$

\section{Sistemas de control}

\subsection{Tipos de sistemas de control de lazo cerrado}

\subsubsection{Control ON-OFF}

En procesos en los que no se requiere un control muy preciso, el control dos-posiciones on/off, puede ser el adecuado. \\

En este tipo de control, el elemento final de control se mueve rápidamente entre una de dos posiciones fijas a la otra, para un valor único de la variable controlada. \\

Un controlador on-off opera sobre la variable manipulada solo cuando la temperatura cruza la temperatura deseada SP. La salida tiene solo dos estados, completamente activado (on) y completamente desactivado (off). Un estado es usado cuando la temperatura está en cualquier lugar sobre el valor deseado y el otro estado es usado cuando la temperatura está en cualquier punto debajo de la temperatura deseada SP. \\

\begin{figure}[H]
    \centering
    \includegraphics[scale=.5]{src/imagenes/anexos/onOff.jpg}
    \caption{Diagrama de un control ON-OFF.}
\end{figure}

Dado que la temperatura debe cruzar la temperatura deseada SP para cambiar el estado de salida, la temperatura del proceso estará continuamente oscilando. \\

Para evitar dañar los dispositivos de control finales un diferencial on-off o histéresis es agregada a la función del controlador. Esta función requiere que la temperatura exceda la temperatura deseada SP en una cierta cantidad antes de que la salida se desconecte nuevamente. La histéresis evitará vibraciones de la salida si el ruido de pico a pico es menor que la histéresis. La cantidad de histéresis determina la variación mínima de temperatura posible.

Los controladores son elementos que se le agregan al sistema original para mejorar sus características de funcionamiento, con el objetivo de satisfacer las especificaciones de diseño tanto en régimen transitorio como en estado estable. \\

La primera forma para modificar las características de respuesta de los sistemas es el ajuste de ganancia (lo que posteriormente se definirá como control proporcional). Sin embargo, aunque por lo general el incremento en ganancia mejora el funcionamiento en estado estable, se produce una pobre respuesta en régimen transitorio y viceversa.

\subsubsection{Compensador de fase}

Un compensador es un componente adicional que es aumentado a un sistema de control para modificar el desempeño en lazo cerrado y compensar por un desempeño deficiente. A diferencia de los controladores, los compensadores pueden ubicarse en cualquier posición del sistema de control. \\

\textbf{Compensador de atraso} \\

La compensación de atraso produce un mejoramiento notable en la precisión en estado estable a costa de aumentar el tiempo de respuesta transitoria. Suprime los efectos de las señales de ruido a altas frecuencias. Aumenta el orden del sistema en 1 (a menos que haya una cancelación entre el cero del compensador y un polo de la función de transferencia en lazo abierto no compensada). \\

La función principal de un compensador de atraso es proporcionar una atenuación en el rango de las frecuencias altas a fin de aportar un margen de fase suficiente al sistema. La característica de atraso de fase no afecta la compensación de atraso. \\

\textbf{Compensador de adelanto} \\

Produce un mejoramiento razonable en la respuesta transitoria y un cambio pequeño en la precisión en estado estable. Puede acentuar los efectos del ruido de alta frecuencia. Aumenta el orden del sistema en 1 (a menos que haya una cancelación entre el cero del compensador y un polo de la función de transferencia en lazo abierto no compensada). \\

\textbf{Compensador de atraso-adelanto} \\

La compensación de atraso-adelanto combina las características de la compensación de adelanto con las de la compensación de atraso. El uso de un compensador de atraso adelanto eleva el orden del sistema en 2 (a menos que haya una cancelación entre el cero, o los ceros, del compensador de atraso adelanto y el polo, o los polos, de la función de transferencia en lazo abierto no compensada).

\subsubsection{Controlador P (proporcional)}

Se dice que un control es de tipo proporcional cuando la salida del controlador $v(t)$ es proporcional al error $e(t)$:

$$v(t) = Kp \cdot e(t)$$

Puesto que la ganancia $Kp$ del controlador es proporcional, ésta puede ajustarse. En general, para pequeñas variaciones de ganancia, aunque se logra un comportamiento aceptable en régimen transitorio, la respuesta de estado estable lleva implícita una magnitud elevada de error. \\

Al tratar de corregir este problema, los incrementos de ganancia mejorarán las características de respuesta de estado estable en detrimento de la respuesta transitoria. Por lo anterior, aunque el control P es fácil de ajustar e implementar, no suele incorporarse a un sistema de control en forma aislada, sino más bien se acompaña de algún otro elemento.

\subsubsection{Controlador D (derivativo)}

Se dice que un control es de tipo derivativo cuando la salida del controlador $v(t)$ es proporcional a la derivada del error $e(t)$:

$$v(t) = Kd \cdot \frac{de(t)}{dt}$$

Donde Kd es la ganancia del control derivativo. La constante Kd puede escribirse en términos de Kp:

$$Kd = Kp \cdot Td$$

donde Td es un factor de proporcionalidad ajustable que indica el tiempo de derivación. \\

El significado de la derivada se relaciona con la velocidad de cambio de la variable dependiente, que en el caso del control derivativo indica que éste responde a la rapidez de cambio del error, lo que produce una corrección importante antes de que el error sea elevado. Además, la acción derivativa es anticipativa, esto es, la acción del controlador se adelanta frente a una tendencia de error (expresado en forma de derivada). Para que el control derivativo llegue a ser de utilidad debe actuar junto con otro tipo de acción de control, ya que, aislado, el control derivativo no responde a errores de estado estable.

\subsubsection{Controlador I (integral)}

Se dice que un control es de tipo integral cuando la salida del controlador $v(t)$ es proporcional a la integral del error $e(t)$:

$$v(t) = Ki \cdot \int e(t) \cdot dt$$

Donde Ki es la ganancia del control integral. En cualquier tipo de controlador, la acción proporcional es la más importante, por lo que la constante Ki puede escribirse en términos de Kp:

$$Ki = \frac{Kp}{Ti}$$

Donde Ti es un factor de proporcionalidad ajustable que indica el tiempo de integración. \\

El control integral tiende a reducir o hacer nulo el error de estado estable, ya que agrega un polo en el origen aumentando el tipo del sistema; sin embargo, dicho comportamiento muestra una tendencia del controlador a sobrecorregir el error. Así, la respuesta del sistema es de forma muy oscilatoria o incluso inestable, debido a la reducción de estabilidad relativa del sistema ocasionada por la adición del polo en el origen por parte del controlador.

\subsubsection{Controlador PI (proporcional-integral)}

Se dice que un control es de tipo proporcional-integral cuando la salida del controlador $v(t)$ es proporcional al error $e(t)$, sumado a una cantidad proporcional a la integral del error $e(t)$:

$$v(t) = Kp \cdot e(t) + \frac{Kp}{Ti} \cdot \int e(t) \cdot dt$$

La ecuación corresponde a un factor proporcional Kp que actúa junto con un cero ubicado en $z = \frac{-1}{Ti}$ (cuya posición es ajustable sobre el eje real a la izquierda del origen) y un polo en el origen.

\subsubsection{Controlador PD (proporcional-derivativo)}

Se dice que un control es de tipo proporcional-derivativo cuando la salida del controlador $v(t)$ es proporcional al error $e(t)$, sumado a una cantidad proporcional a la derivada del error $e(t)$:

$$v(t) = Kp \cdot e(t) + Kp \cdot Td \cdot \frac{de(t)}{dt}$$

La ecuación indica un factor proporcional Kp Td, que actúa junto con un cero $z = \frac{-1}{Td}$, cuya posición es ajustable en el eje real. \\

La ecuación indica un factor proporcional Kp Td que actúa junto con un par de ceros (distintos, repetidos o complejos, cuya posición es ajustable en el plano s) y un polo en el origen.

\subsubsection{Controlador PID (proporcional-integral-derivativo)}

Se dice que un control es de tipo proporcional-integral-derivativo cuando la salida del controlador $v(t)$ es proporcional al error $e(t)$, sumado a una cantidad proporcional a la integral del error $e(t)$ más una cantidad proporcional a la derivada del error $e(t)$:

$$v(t) = Kp \cdot e(t) + \frac{Kp}{Ti} \cdot \int e(t) \cdot dt + Kp \cdot Td \cdot \frac{de(t)}{dt}$$

\subsubsection{Características de los tipos de controladores}

Como conclusión, se enumeran las principales características de los diferentes tipos de controladores: P, PI, PD y PID. \\

\textbf{Control proporcional}:

\begin{itemize}
    \item{El tiempo de elevación experimenta una pequeña reducción.}
    \item{El máximo pico de sobreimpulso se incrementa.}
    \item{El amortiguamiento se reduce.}
    \item{El tiempo de asentamiento cambia en pequeña proporción.}
    \item{El error de estado estable disminuye con incrementos de ganancia.}
    \item{El tipo de sistema permanece igual.}
\end{itemize}

\textbf{Control proporcional-integral:}

\begin{itemize}
    \item{El amortiguamiento se reduce.}
    \item{El máximo pico de sobreimpulso se incrementa.}
    \item{Decrece el tiempo de elevación.}
    \item{El tipo de sistema se incrementa en una unidad.}
    \item{El error de estado estable mejora por el incremento del tipo de sistema.}
\end{itemize}

\textbf{Control proporcional-derivativo:}

\begin{itemize}
    \item{El amortiguamiento se incrementa.}
    \item{El máximo pico de sobreimpulso se reduce.}
    \item{El tiempo de elevación experimenta pequeños cambios.}
    \item{Se mejoran el margen de ganancia y el margen de fase.}
    \item{El error de estado estable presenta pequeños cambios.}
    \item{El tipo de sistema permanece igual.}
\end{itemize}

\textbf{Control proporcional-integral-derivativo:} \\

Este tipo de controlador contiene las mejores características del control proporcional-derivativo y del control proporcional-integral.

\subsection{Efectos de añadir polos y ceros al sistema}

\textbf{Efecto de añadir polos:} \\

El incremento en el número de polos en un sistema ocasiona que el lugar geométrico de raíces se desplace hacia la derecha del eje $j \omega$, lo que reduce la estabilidad relativa del sistema o, en algunos casos, lo hace inestable. \\

\textbf{Efecto de añadir ceros:} \\

Incorporar ceros en un sistema produce que el lugar geométrico de raíces se desplace hacia el semiplano izquierdo, lo que hace estable o más estable al sistema. \\

En términos generales, el diseño de los controladores se enfoca en la adición de ceros para mejorar la respuesta transitoria, así como la colocación de un polo en el origen para corregir el comportamiento de estado estable del sistema.

\subsection{Diagrama de Bode}

Un diagrama de Bode es una representación gráfica que sirve para caracterizar la respuesta en frecuencia de un sistema. Normalmente consta de dos gráficas separadas, una que corresponde con la magnitud de dicha función y otra que corresponde con la fase. Recibe su nombre del científico estadounidense que lo desarrolló, Hendrik Wade Bode. \\

La respuesta en amplitud y en fase de los diagramas de Bode no pueden por lo general cambiarse de forma independiente: cambiar la ganancia implica cambiar también desfase y viceversa. En sistemas de fase mínima (aquellos que tanto su sistema inverso como ellos mismos son causales y estables) se puede obtener uno a partir del otro mediante la transformada de Hilbert.