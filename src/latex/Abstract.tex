\chapter*{Abstract}
\addcontentsline{toc}{chapter}{Abstract}

\pagestyle{empty}

Se plantea el desarrollo de un sistema de control de oxígeno disuelto en la pileta de lodos activos de la planta de tratamiento de aguas residuales de la empresa de EGGER en Concordia, ubicada en el parque industrial. \\

Actualmente, en la industria, el proceso de aireación no posee un sistema de control. Esto quiere decir que los sopladores que intervienen están funcionando a potencia máxima todo el tiempo. Lo que supone una serie de limitaciones que son analizadas en el proyecto y se le dan solución de la manera más práctica posible. \\

Se construye un prototipo del sistema para poner a prueba distintos escenarios a los cuales estaría expuesto el controlador ya implementado. Para ello se realizó un desarrollo simplicado que de la experiencia de estar interviniendo en el proceso real.