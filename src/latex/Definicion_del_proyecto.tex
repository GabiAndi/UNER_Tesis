\chapter{Definición del proyecto}

\pagestyle{empty}

\newpage

\pagestyle{fancy}

\section{Objetivos}

\subsection{Objetivo general}

Desarrollar un sistema de control basado en tecnologías de Industria 4.0, para el proceso de aireación de la pileta de lodos activos ubicada en la planta de tratamiento de efluentes de EGGER.

\subsection{Objetivos específicos}

Para lograr lo anteriormente mencionado se deberá:

\begin{itemize}
    \item{Analizar y comprender el proceso de tratamiento de aguas residuales de EGGER.}
    \item{Definir la infraestructura y el control con el que cuenta actualmente la empresa en la parte de tratamiento secundario.}
    \item{Determinar la variable a controlar y las variables críticas del proceso de lodos activos.}
    \item{Desarrollar un prototipo para realizar pruebas y ensayos buscando demostrar la viabilidad del sistema de control.}
    \item{Analizar el desempeño energético del sistema de control.}
\end{itemize}

\section{Alcance}

El proyecto abarca el diseño y desarrollo de un sistema de control con las características anteriormente nombradas, incluyendo:

\begin{itemize}
    \item{Planos, esquemas y cálculos del diseño.}
    \item{Desarrollo y construcción de un prototipo del sistema de control a baja escala, simulando la interacción entre la pileta y el controlador.}
    \item{Desarrollo del software que ejecutará el controlador y el SCADA.}
    \item{Desarrollo del software encargado de la simulación del comportamiento del sistema para el prototipo.}
    \item{Configuración de la red operativa del prototipo (IP, puertos, firewall, servicios, etc.).}
    \item{Cálculo de la inversión para el sistema de control y el prototipo.}
\end{itemize}

Con esto se da solución a la problemática inicial, pero también se busca introducir tecnologías de industria 4.0 al proceso, lo que traerá una serie de futuras mejoras que le darán valor agregado al producto.

\section{Limitaciones del alcance}

El proyecto, debido a la complejidad de desarrollo, no incluirá una comunicación más allá de TCP/IP, como lo pueden ser profinet, profibus, modbus, etc.

El análisis de la fatiga de los caños sometidos al golpe de ariete generado por los sopladores, fenómenos químicos, consideraciones químicas o físicas, más allá de la intervención del sistema de control, no son contemplados en lo absoluto.