\chapter{Caracterización}

\pagestyle{empty}

\newpage

\pagestyle{fancy}

\section{Objetivos}

\subsection{Objetivos generales}

Se busca desarrollar un sistema de control basado en tecnologías de Industria 4.0 para el proceso de aireación de la pileta de lodos activos. Con este proyecto se pretende tener un control inteligente y adaptable sobre la cantidad de oxígeno disuelto de la pileta de forma permanente. Automatizar el procedimiento de arranque y parada de los sopladores, y gestionar la rotación de los mismos.

\subsection{Objetivos específicos}

Gracias a la incorporación de lo anteriormente mencionado, con este sistema implementado se obtendrá:

\begin{itemize}
    \item{Desbloqueo de un potencial ahorro energético.}
    \item{Control dinámico en el suministro de OD.}
    \item{Métricas en tiempo real de todo el sistema (interconexión de sistemas).}
    \item{Almacenamiento de todas las métricas producidas para un posterior análisis (bigdata).}
    \item{Control a distancia del proceso (IoT).}
    \item{Desbloqueo de potenciales mejoras (análisis de datos).}
    \item{Posibidad de ampliación del sistema de control a otras areas del proceso (escalabilidad).}
\end{itemize}

\section{Alcance}

El proyecto abarca el diseño y desarrollo de un sistema de control con las carácteristicas anteriormente nombradas (planos, esquemas y cálculos), el diseño, desarrollo y construcción de un prototipo del sistema de control (planos, esquemas y cálculos), a baja escala y simulando la interacción de la pileta y el controlador, el software que correrá el controlador y el ESCADA que correrá una computadora de escritorio, celular o el propio HMI en planta (prototipo unicamente), el software encargado de la simulación del comportamiento del sistema (prototipo unicamente), la configuración de la red operativa del prototipo (IPs, puertos, firewall, servicios, etc), cálculo de la inversión para el sistema completo y el prototipo.

\section{Limitaciones del alcance}

El proyecto, debido a la complejidad de desarrollo no incluirá una comunicación más alla de TCP/IP, como lo pueden ser profinet, profibus, vss, modbus, etc. \\

El análisis de la fatiga de los caños sometidos al golpe de ariete generado por los sopladores, fenomenos químicos, consideraciones químicas o físicas mas alla de la intervención del sistema de control no son contemplados en lo absoluto.