\chapter{Conclusiones}

\pagestyle{empty}

\newpage

\pagestyle{fancy}

\section{Análisis técnico-económico}

La factibilidad técnica-económica busca demostrar la viabilidad o no de un proyecto, tal como lo establece \href{https://econforesyproyec.files.wordpress.com/2014/11/evaluacion-de-proyectos-gabriel-baca-urbina-corregido.pdf}{Baca Urbina (2001)} son herramientas del área financiera que permiten la toma de decisiones acertadas para invertir en ampliaciones de empresas o el establecimiento de nuevas empresas que conlleven al éxito.

Partiendo de estas premisas, se realizaron los estudios necesarios para determinar la viabilidad del proyecto que represente no sólo la mejora en cuanto eficiencia del proceso sino también la oportunidad de establecer planes que generen un desarrollo escalable y capaz de adaptarse a la infraestructura que ya se encuentra en funcionamiento.

Por lo cual, se puede concluir en base a los estudios realizados, que existe un amplio potencial de mejora. Según el criterio de análisis aplicado al proceso, y validado por el prototipo construido, el sistema simulado reacciona de manera esperada ante la intervención de un controlador.

En el estudio económico se realizaron las investigaciones para determinar los costos de inversión inicial cuyo monto es de 5.\$107.888,66ARS o \$45.395,38USD.

El análisis de sensibilidad determinará los escenarios más pesimistas y más optimistas que puede presentar el proyecto, para realizarlo se utilizaron promedios de ahorro del 20\%, 30\% y 40\%.

Para el mejor de los casos, el tiempo de recupero de la inversión es de casi 4 años. Esto podría ser bastante tiempo, sin embargo, solo se estaría contemplando el ahorro energético del proceso, sin tener en cuenta las mejoras respecto a la facilidad de uso y la incorporación de nuevas tecnológicas.

\textbf{Incluir análisis del retorno de la inversión.}

Los resultados obtenidos permiten la elaboración de una propuesta que, de ser aplicada o tomada en cuenta, le otorga a la empresa la posibilidad de controlar y mejorar el tratamiento secundario de lodos activos, proceso que posee un consumo energético elevado. Además, se proporciona una infraestructura tecnológica de industria 4.0 con las ventajas de: tener un control a demanda, a distancia, orientado a la conectividad y captura de datos. Lo que abre la puerta al mundo del Big Data y sus diferentes aplicaciones orientadas a mejorar la toma de decisiones, predecir fallas y hasta aprendizaje automático.

\section{Prototipado}

Con la aplicación de la metodología propuesta en el desarrollo, se pudo comprobar de manera práctica que es indispensable visualizar todas las posibilidades o decisiones dentro del proceso, para poder escoger de la mejor forma pensando en minimizar las correcciones en un futuro. Siendo que tomando este camino el proceso utiliza mayor cantidad de recursos tanto de tiempo cómo económicos, pero a la larga termina siendo más conveniente, ya que los recursos utilizados en esta etapa reducen la posibilidad de errores futuros lo que se traduce en un ahorro sustancial de los recursos disponibles para el proyecto.

Pero en el caso de que sea necesario realizar una modificación en el producto que se está desarrollando, se cuenta con todos los antecedentes necesarios para tomar una decisión enfocándose al punto donde se encuentra el error.

De esta forma se puede diferenciar un proceso que tiene como desventaja el tiempo de fabricación que toman las piezas, pero que da la posibilidad de poder variar aspectos tales como resistencia y flexibilidad manejando la geometría o el relleno de las piezas, con la particularidad de que se utiliza el mismo material.

Llegamos a la conclusión que para lograr que un sistema sea correcto y que tenga todo lo que el cliente pide se tiene que comenzar con un prototipo que este te va dando los detalles buenos y malos del sistema y así el cliente con un prototipo del sistema puede hacer modificaciones de lo que desea en su sistema sin tener que modificar el sistema ya instalado.

\section{Metodologías ágiles}

En los últimos 10 años se ha ido imponiendo el uso del Design Thinking como metodología para el diseño de productos y servicios. Las empresas entienden la importancia de hacer colaborar a sus equipos a la hora de innovar, saben que el Design Thinking es una poderosa metodología para asegurar el impacto de las innovaciones en el negocio.

El Design Thinking persigue incorporar en el proceso de diseño tanto la vertiente técnica como la vertiente humana del usuario al que iba dirigido el diseño. Esta es una metodología muy abierta que se puede aplicar a multitud de problemas a los que nos podemos enfrentar. Más que una herramienta de diseño es una metodología de pensamiento en la que la creatividad, y el trabajo multidisciplinar de equipo facilita la identificación de alternativas que en un principio no son aparentes.

El Design Thinking es una metodología muy versátil que nos proporciona una manera alternativa de afrontar cualquier tipo de problema. Su uso ha sido adoptado por las empresas que están consiguiendo mayores éxitos desde la innovación y esto se está traduciendo en mayores beneficios y está transformando, cada vez más, la manera de trabajo.

\section{Industria 4.0 en Argentina}

\textbf{Nota:} \textit{Para la elaboración de este resumen, se consulto al estudio \cite{I40AR} sobre la situación actual de la Industria 4.0 en el país.}

La nueva era tecnológica sin dudas está impactando mundialmente. La inteligencia artificial sumada al desarrollo de otros avances tecnológicos, generan consecuencias en todas las actividades humanas y producción de bienes y servicios.

Aunque crecientemente popular, el término Industria 4.0 es relativamente nuevo. En términos muy generales, la I4.0 se presenta como la digitalización de la industria a través de la convergencia de la información y la producción, así como la del servicio y la fabricación.

Las modificaciones en curso son tan importantes que los países industrializados han lanzado sus propios planes para impulsar la I4.0. Frente a eso, los países subdesarrollados no han encarado esta agenda de reflexión salvo en contadísimas excepciones.

Claro está que Argentina es un país que se encuentra con diversos problemas económicos, por lo que se halla un poco atrasado en cuanto a la innovación. La opción por apostar en invertir en tecnologías inteligentes no se encuentra dentro de las primeras opciones para un gran porcentaje de ejecutivos, dado que buscan apostar por el corto plazo.