\chapter{Identificación del problema}

\pagestyle{empty}

\newpage

\pagestyle{fancy}

\section{Planta de tratamientos}

En Concordia, Entre Ríos, existe una industria que se dedica a la elaboración de productos a base de madera. La planta es un complejo productivo donde se fabrican y distribuyen tableros de fibra de madera (MDF), tableros de aglomerado (PB), Melamina, Molduras y Revestimientos.

La planta de tratamiento, consiste en un conjunto de instalaciones y equipos destinados a tratar los efluentes líquidos industriales generados en las dos líneas de producción de tableros de fibra de madera (MDF). La materia prima en forma de chips húmedos, antes de ser desfibrada, sufre un proceso de vaporización a baja presión. Luego es comprimida por un tornillo cónico donde se produce una corriente líquida con alta carga de materia orgánica en suspensión. La misma se compone de los extractos típicos de la madera, su propia humedad y el vapor de agua condensado que fuera utilizado para el calentamiento.

Por lo tanto, el objetivo del tratamiento de las aguas residuales, es producir un efluente reutilizable y un residuo sólido para su posterior uso.

\begin{figure}[H]
    \centering
    \includegraphics[scale=.5]{src/imagenes/tratamiento/diagramaDePlantaDeEfluentes.png}
    \caption{Diagrama de flujo del proceso de tratamiento}
    \label{fig:digramaDeTratamientoDeEfluentes}
\end{figure}

El proceso de tratamiento posee cuatro etapas:

\begin{enumerate}
    \item{\textbf{Preliminar:} tratamiento de sólidos gruesos. Filtrado físico por rejillas, etc.}
    \item{\textbf{Primario:} tratamiento de sólidos en suspensión. Añadido de químicos floculantes.}
    \item{\textbf{Secundario:} tratamiento biológico. Lodos activos.}
    \item{\textbf{Terciario:} tratamiento final. Desinfección y filtrado ultrafino.}
\end{enumerate}

En la figura \ref{fig:digramaDeTratamientoDeEfluentes} se ve de manera simplicada cada etapa individual del proceso (para más información ver anexo \ref{anexo:tratamientoDeEfluentes}). Sin embargo, es necesario conocer un poco más sobre el tratamiento secundario de tipo biológico. Es por esto que recomendamos leer la sección \ref{anexo:tratamientoSecundario} del anexo.

\section{Planteo de la problemática}

El proceso de tratamiento de efluentes opera de manera ininterrumpida las 24 hs del día, los 365 días del año. Lo que supone que todo el sistema debe estar a punto para evitar contratiempos, o paradas innecesarias que perjudiquen directamente a los demás procesos que dependen de este.

Es lógico asumir que un proceso que opera de manera ininterrumpida, posea grandes requerimientos energéticos. De hecho, la etapa secundaria de una planta de tratamiento de efluentes es responsable de más 90\% del consumo total de la misma.

El tratamiento secundario de tipo biológico, posee dos sopladores de \href{https://es.wikipedia.org/wiki/Compresor_volum%C3%A9trico_tipo_Roots}{lóbulos rotativos}, impulsados cada uno de ellos por un motor trifásico de 75 HP (55 kW).

\begin{center}
    \includegraphics[scale=.5]{src/imagenes/sistema/blowerRoots.png}
\end{center}

\begin{enumerate}
    \item[1.]{Rotor 1.}
    \item[2.]{Cuerpo de la bomba.} 
    \item[3.]{Rotor 2.}
    \item[a.]{Admición.}
    \item[b.]{Compresión.}
    \item[c.]{Expulsión forzada.}
\end{enumerate}

Estos motores son los causantes del elevado consumo de electricidad de toda la planta, y no siempre es necesario que funcionen a potencia nominal \footnote{Actualmente los sopladores \textbf{no} poseen un control de velocidad a demanda, y el proceso de puesta en marcha de los motores se realiza mediante un arranque directo.}, ya que depende del estado actual del efluente cuanto aire requiere que estén soplando. Además, como ningún motor sería capaz de soportar tanto tiempo ininterrumpido de operación, ambos tienen una rotación de uso de 15 días aproximadamente, esta conmutación se efectúa de manera manual y es causante de fallas estructurales en la cañería de transporte de aire (ruptura por golpe de ariete, fatiga, etc.).

Los dispositivos actuales que controlan el proceso (PLC, HMI, etc.) no tienen comunicación entre ellos más allá de la captura de datos. Las válvulas son operadas manualmente, lo que demanda la intervención en el proceso en cada regulación y cambio de sopladores.

A todo esto, se suma el hecho de disponer de un sistema \textit{bobo} de control del proceso, es decir, un sistema sin análisis de los datos producidos, sin conectividad con el exterior de la planta de tratamiento y sin la de toma de decisiones respecto a las variables que intervienen.

\section{Justificación}

La intervención manual del operario para cada ajuste del sistema, y el poco control sobre el mismo, produce:

\begin{itemize}
    \item{Riesgos de daños mecánicos a las tuberías en la rotación de operación de sopladores.}
    \item{Consumo elevado del sistema.}
    \item{Poco control en los requerimientos del oxígeno disuelto (OD).}
    \item{Falta de análisis de las métricas producidas.}
    \item{Muy bajo margen de mejora en la eficiencia total del proceso.}
\end{itemize}

Si ocurre una falla crítica en la que el operario tenga que intervenir rápidamente, y esto sucede durante sus horas de ausencia. El personal del área carece de medios que lo notifiquen del hecho, por lo que la acción correctiva sucederá transcurrido un cierto tiempo desde que se dio lugar a la falla.

Todos estos inconvenientes hacen necesario un sistema que controle el proceso y proporcione métricas de su funcionamiento. Además de tener la capacidad de monitorizar su estado a distancia y tomar acciones correctivas para minimizar los daños ocasionados ante una eventual falla del sistema.