\chapter{Diseño e Ingeniería}

\pagestyle{empty}

\newpage

\pagestyle{fancy}

\section{Planificación}

Para poder llevar a cabo el desarrollo del proyecto debemos plantearnos una serie de objetivos en un tiempo determinado. Todo esto ayudará a definir las distintas etapas de investigación y desarrollo para lograr un orden y cumplir con los plazos de entrega.

\subsection{Metodologías de trabajo}

\subsubsection{Kanban}

Kanban es una forma de ayudar a los equipos a encontrar un equilibrio entre el trabajo que necesitan hacer y la disponibilidad de cada miembro del equipo. La metodología Kanban se basa en una filosofía centrada en la mejora continua, donde las tareas se representan en una lista de acciones pendientes en un flujo de trabajo constante. \\

La metodología Kanban se implementa por medio de tableros Kanban. Se trata de un método visual de gestión de proyectos que permite a los equipos visualizar sus flujos de trabajo y la carga de trabajo. En un tablero Kanban, el trabajo se muestra en un proyecto en forma de tablero organizado por columnas. \\

Para esto se utilizó el panel de Project Management de GitHub que es gratis y muy completo, además de integrarse con el desarrollo de software.

\begin{figure}[H]
    \centering
    \includegraphics[scale=.28]{src/imagenes/metodologias/kanban.png}
    \caption{Panel de gestión de proyecto en GitHub.}
\end{figure}

\subsubsection{Waterfall}

Esta metodología es una aproximación lineal al desarrollo y ha sido la más utilizada durante los últimos 30 años. La secuencia que sigue este método está compuesta de las siguientes fases:

\begin{itemize}
    \item{Captura y documentación de requisitos.}
    \item{Diseño.}
    \item{Desarrollo.}
    \item{Test.}
    \item{Corrección de errores y ajustes finales.}
    \item{Puesta en producción.}
\end{itemize}

En un desarrollo waterfall puro, cada una de estas fases representan una etapa diferenciada en el desarrollo del producto final. Cada una de estas etapas deben darse por concluidas antes de comenzar con la siguiente. Además, entre cada una de ellas, generalmente tenemos un hito bloqueante que nos impide avanzar a la siguiente si este no se libera adecuadamente.

\subsection{Tareas a realizar}

El desarrollo de este proyecto se divide en 4 grandes etapas:

\begin{enumerate}
    \item{Desarrollo del modelo del sistema.}
    \item{Diseño del hardware de control.}
    \begin{enumerate}
        \item{Actuadores de potencia.}
        \item{Electrónica del controlador.}
        \item{Electrónica transductora.}
        \item{Conexiones clableadas.}
    \end{enumerate}
    \item{Diseño del software de control.}
    \begin{enumerate}
        \item{Controlador.}
        \item{Pantallas de visualización y configuración.}
        \item{Simulación de la pileta.}
        \item{Infraestructura de red.}
        \item{Configuración del servidor de métricas.}
    \end{enumerate}
    \item{Elavoración del prototipo.}
    \begin{enumerate}
        \item{Limitaciones.}
        \item{Contrucción.}
    \end{enumerate}
\end{enumerate}

Todas las actividades son desarroladas secuencialmente utilizando la metodología de waterfall. Unicamente en la etapa de diseño de software se utilizará un enfoque pararelo tipo scrum.

\subsection{Tiempos de desarrollo}

Llevar cabo un desarrollo completo del sistema se extiende un poco, mas si tenemos en cuenta que lo realizan dos Ingenieros Junior. Con esto en cuenta, consideramos los siguientes plazos: \\

\pendiente{Gantt de los tiempos de desarrollo}Aquí va el Gantt. \\

Resumiendo el Gantt, se estipula un tiempo de desarrollo de 10 meses.

\section{Diseño y desarrollo}

\subsection{Modelado del sistema}

\agregar{Modelado del sistema}

\subsection{Diseño eléctrico}

Diseño de la etapa de potencia del sistema de control. Selección de componentes, dimensionamiento de conductores, consideraciones técnicas, etc.

\subsection{Diseño electrónico}

Diseño de la etapa de control de la parte de potencia. Selección de componentes, ubicación, cableado, consideraciones técnicas, etc.

\subsection{Diseño de software}

Diseño y configuración de las tecnologías de Software utilizadas en el sistema de control. Breve reflexión sobre el Software libre y el movimiento OpenSource.

\section{Prototipado}

\subsection{Limitaciones}

Porque es un prototipo, que diferencias tiene respecto al controlador final.

\subsection{Diseño y desarrollo}

Basado en la parte de diseño de ingeniería, que componentes y tecnología implementamos.

\subsection{Ensayos y conclusiones}

Una vez terminada la construcción del prototipo, descripción de las pruebas efectuadas para comprobar la fidelidad y la factibilidad de despliegue a gran escala del sistema real. Conclusiones generadas a partir de dichos ensayos.