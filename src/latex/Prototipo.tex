\chapter{Desarrollo del prototipo}

\pagestyle{empty}

\section*{Descripción del capítulo}

Para que lo desarrollado anteriormente tenga sentido alguno, se construyó un prototipo a baja escala del sistema de control en las instalaciones de la FCAL. Para así poder ver la configuración e implementación de los software que estarían corriendo en el proceso, la parte de potencia, y del control de potencia. \\
	
Todo esto fue diseñado para una demostración durante un periodo de tiempo breve, debido a que el motor no tiene una refrigeración independiente.

\newpage

\pagestyle{fancy}

\section{Configuración de la Raspberry Pi}
	
La configuración de la Raspberry Pi es la que se detallo en la sección de instalación del sistema operativo, en el capítulo de Software.

\section{Configuración del entorno de desarrollo}

Tanto la construcción como la configuración de las herramientas de programación y desarrollo fueron las mismas que se comentaron en el capítulo de Software.

\section{Armado del control de potencia}

El control de potencia se conforma por la Raspberry pi (controlador) y el conjunto motor-variador (potencia). \\

El control de la velocidad del motor se realiza mediante la comunicación entre la Raspberry Pi 3B+ y el variador de velocidad por las entradas y salidas digitales. El variador de velocidad utilizado es el Siemens Micromaster 440.

\subsection{Variador de velocidad}

La serie Micromaster 440 es una gama de convertidores de frecuencia para modificar la velocidad de motores trifásicos. Los distintos modelos disponibles abarcan un rango de potencias desde 120 W para entrada monofásica hasta 75 kW con entrada trifásica. \\

Los convertidores están controlados por microprocesador y utilizan tecnología IGBT (Insulated Gate BipoIar Transistor). Esto los hace fiables y versátiles. Un método especial de modulación por ancho de impulsos con frecuencia de pulsación seleccionable permite un funcionamiento silencioso del motor. Extensas funciones de protección ofrecen una protección tanto del convertidor como del motor. 

\begin{figure}[H]
    \centering
    \includegraphics[angle=-90,scale=.06]{src/imagenes/prototipo/variador.jpg}
\end{figure}

\textbf{Características:}

\begin{itemize}
    \item{Amplio número de parámetros que permite la configuración de una gama extensa de aplicaciones.}
    \item{Relés de salida.}
    \item{Salidas analógicas (0-20 mA).}
    \item{6 entradas digitales NPN/PNP aisladas y conmutables.}
    \item{2 entradas analógicas.}
    \item{Las 2 entradas analógicas se pueden utilizar como la 7 y 8 entrada digital.}
    \item{Opciones externas para comunicaciones por PC, panel BOP (Basic Operator Panel), panel AOP (Advanced Operator Panel) y módulo de comunicación PROFIBUS}
    \item{Control vectorial sin sensores (sensorless vector control).}
    \item{Control de flujo corriente FCC (flux current control) para una mejora de la respuesta dinámica y control del motor.}
    \item{Limitación rápida de corriente FCL (fast current limitation) para funcionamiento libre de disparos intempestivos}
\end{itemize}

\subsection{Motor}

El motor elegido es un CORRADI MTA-80b/2 de 1.1kW que se encontraba en la facultad. El único propósito del motor es hacer visible el cambio de velocidad.

\begin{figure}[H]
    \centering
    \includegraphics[angle=-180,scale=.06]{src/imagenes/prototipo/motor.jpg}
\end{figure}

\subsection{Raspberry Pi 3B+}

Se eligió una Rasperry Pi 3B+ porque se adapta perfectamente a las necesidades de IoT de este proyecto. Ademas de la extensa documentación que se encuentra en internet.

\section{Servidor de métricas}

La configuración del servidor de métricas es el que se detallo en el capítulo de Software.

\section{Pruebas}

Cuando se termino de realizar todas las conexiones de entre el motor, el variador, la Raspberry y el servidor de métricas, se procede 

\begin{figure}[H]
    \centering
    \includegraphics[angle=-90,scale=.06]{src/imagenes/prototipo/sistemaCompleto.jpg}
\end{figure}

\begin{figure}[H]
    \centering
    \includegraphics[angle=-90,scale=.06]{src/imagenes/prototipo/hmi.jpg}
\end{figure}