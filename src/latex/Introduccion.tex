\chapter{Introducción}

\pagestyle{empty}

\newpage

\pagestyle{fancy}

\section{Estructura del documento}

El documento fue dividido en múltiples capítulos que corresponden a las etapas del proyecto. A continuación se describe el contenido de cada uno.
	
\section*{Identificación del problema}

Este capítulo introduce sobre la problemática de la empresa y como se abordó una idea que busca ser implementable a corto plazo, pero que genere soluciones y mejoras a largo plazo. \\

Se describe detalladamente el tipo de proceso con el cual se esta trabajando (planta de tratamiento biológico), realizando las observaciones necesarias para poder entender como trabaja este tipo de sistemas. \\

Se utilizaron la mayoría de habilitadores tecnológicos que impulsan el desarrollo 4.0 en industria para abordar esta solución.

\section*{Marco teórico}

Se desarrollan la mayoría de conceptos teóricos y técnicos que se tuvieron en cuenta para la etapa de investigación y desarrollo.

\section*{Proyecto}

Aquí se realiza un análisis de los distintos aspectos a la hora de formular un proyecto de automatización para industria, además de estudiar en profundidad los aspectos a tener en cuenta a la hora de trabajar en un tratamiento biológico. \\

La primer etapa de diseño constructivo da la pauta hacia donde es conveniente enfocar los esfuerzos del proyecto.

\section*{Diseño del hardware}

Ya con las pautas de construcción establecidas, se realiza un estudio de componentes físicos necesarios para lograr intervenir en el proceso, pudiendo así realizar todas las mediciones y actuaciones que permitan realizan un control del sistema de tratamiento de tipo biológico.

\section*{Diseño del software}

Abarca todo lo necesario a la hora de diseñar e implementar un software de control a la industria. Cubre aspectos como: confidencialidad de datos, protección de datos personales, buenas prácticas a la hora del desarrollo de software, buenas prácticas a la hora del diseño de interfaces de usuario orientadas a operadores, etc.

\section*{Tecnología y software utilizados}

Descripción breve sobre todo el software y las tecnologías utilizadas en el proyecto. Se utilizó en la medida de lo posible proyectos opensource para fomentar el uso de estos programas.

\section*{Desarrollo del prototipo}

Luego de la fase de diseño del sistema de control real, surge la necesidad de construir un prototipo a muy baja escala, el cual pueda mostrar casi con la misma esencia el comportamiento de este tipo de aplicaciones orientados a la industria 4.0. \\

La fase de desarrollo muestra con imágenes el proceso de montaje en la Facultad De Ciencias de la Alimentación.

\section*{Conclusiones}

Reflexiones acerca del desarrollo del proyecto final para conseguir el título de grado. Problemas encontrados y como se solucionaron, comportamientos inesperados y como se corregirían, etc.