\pagestyle{empty}

\section*{Introducción}

En un mundo donde el coste y consumo de la energía está creciendo y se evidencia el impacto de la era industrial sobre el cambio climático, mejorar la eficiencia de los procesos marca la diferencia. Estas mejoras, no consisten simplemente en reducir el consumo energético, sino en utilizar la energía de manera más racional conforme a su condición de escasez y al carácter no renovable de algunas de sus fuentes de generación. En la actualidad, los sistemas de control se postulan como un aliado para que las empresas consigan este objetivo. La utilización de herramientas digitales, hace posible un control en tiempo real del proceso, mejorando sus condiciones de operación y estabilidad, lo que previene picos de consumo.

\section*{Estructura del documento}

El documento fue dividido en múltiples capítulos que corresponden a las etapas del proyecto. A continuación se describe el contenido de cada uno.
	
\subsection*{Identificación del problema}

Este capítulo tiene como objetivo conocer la problemática de la empresa y explicar de manera resumida, el proceso biológico que interviene en la planta de tratamiento de efluentes, realizando las observaciones necesarias para poder entender como trabaja este sistema.

\subsection*{Definición del proyecto}

El análisis de los requerimientos, la definición de los objetivos, el alcance y las limitaciones del alcance, son desarrollados en este capítulo.

\subsection*{Diseño conceptual de la solución}

En este capítulo se aborda de una manera mas técnica y económica, el diseño de la solución. Se propone una metodología de trabajo basada en soluciones alternativas de control para este tipo de sistemas, ya sea comerciales o desarrollos personalizados.

\subsection*{Diseño de ingeniería}

La planificación y el diseño detallado del sistema de control se encuentran en este capítulo. \\

La etapa de planificación abarca: 

\begin{itemize}
    \item{Estimación de tiempos de desarrollo.}
    \item{Análisis económico inicial del proyecto.}
    \item{Análisis de impacto.}
\end{itemize}

La etapa de diseño del sistema de control abarca: 

\begin{itemize}
    \item{Diseño matemático.}
    \item{Diseño eléctrico.}
    \item{Diseño electrónico.}
    \item{Diseño de software.}
\end{itemize}

\subsection*{Diseño y construcción del prototipo}

En este capítulo de describe el proceso de diseño y construcción del prototipo. Al no disponer del sistema real a controlar para realizar las pruebas, se opta por realizar una simulación del comportamiento del mismo, todo esto, así como también los criterios utilizados para su diseño, son descriptos aquí.

\subsection*{Conclusiones del proyecto}

Culminado con el desarrollo, se expone los resultados obtenidos de todo el proceso y variaciones en la estimación de los cálculos técnicos y economicos.