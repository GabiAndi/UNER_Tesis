\chapter{Introducción}

\pagestyle{empty}

En un mundo donde el coste y consumo de la energía está creciendo y se evidencia el impacto de la era industrial sobre el cambio climático, mejorar la eficiencia de los procesos marca la diferencia. Estas mejoras, no consisten simplemente en reducir el consumo energético, sino en utilizar la energía de manera más racional conforme a su condición de escasez y al carácter no renovable de algunas de sus fuentes de generación. En la actualidad, los sistemas de control se postulan como un aliado para que las empresas consigan este objetivo. La utilización de herramientas digitales, hace posible un control en tiempo real del proceso, mejorando sus condiciones de operación y estabilidad, lo que previene picos de consumo.

\newpage

\pagestyle{fancy}

\section*{Estructura del documento}

El documento fue dividido en múltiples capítulos que corresponden a las etapas del proyecto. A continuación se describe el contenido de cada uno.
	
\subsection*{Identificación del problema}

Este capítulo tiene como objetivo conocer la problemática de la empresa y explicar de manera resumida, el proceso biológico que interviene en la planta de tratamiento de efluentes, realizando las observaciones necesarias para poder entender como trabaja este sistema. \\

Se expone la idea de solución basada en la problemática actual del sistema.

\subsection*{Definición del proyecto}

El análisis de los requerimientos, la definición de los objetivos, el alcance y las limitaciones del alcance son desarrollados en este capítulo.

\subsection*{Diseño conceptual de la solución}

En este capítulo se aborda de una manera mas técnica y económica, el diseño de la solución. Se propone una metodología de trabajo basada en soluciones alternativas de control para este tipo de sistemas, ya sea comerciales o desarrollos personalizados. El análisis económico inicial, el análisis de impacto y la metodología planteada también se describen aquí.

\subsection*{Diseño e ingeniería}

La planificación y el diseño detallado del sistema de control, y la contrucción de un prototipo se encuentran en este capítulo. \\

La etapa de planificación abarca: 

\begin{itemize}
    \item{Estimación de tiempos de desarrollo.}
    \item{Análisis económico inicial del proyecto.}
    \item{Análisis de impacto.}
\end{itemize}

La etapa de diseño del sistema de control abarca: 

\begin{itemize}
    \item{Diseño matemático.}
    \item{Diseño eléctrico.}
    \item{Diseño electrónico.}
    \item{Diseño de software.}
\end{itemize}

En la etapa de prototipado se desarrolla sobre: 

\begin{itemize}
    \item{Limitaciones del prototipo.}
    \item{Diseño y desarrollo.}
    \item{Ensayos y conclusiones.}
\end{itemize}

Por último en la sección de escalado al proceso en la industria, se determina un plan de acción y varias consideraciones para poder implementar el sistema real.

\subsection*{Evaluación y conclusiones}

Culminado con el desarrollo, se expone los resultados obtenidos de todo el proceso y variaciones en la estimación de los cálculos técnicos y economicos.